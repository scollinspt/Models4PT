\documentclass[12pt,letterpaper]{article}
\usepackage{ucs}
\usepackage[utf8x]{inputenc}
%\renewcommand{\baselinestretch}{2}

\usepackage{hyperref}
\usepackage{mathptmx}
\usepackage{fullpage}

\raggedright
\urlstyle{rm}

\title{DAGitty: A Graphical Tool for Analyzing Causal Diagrams}
\author{Johannes Textor \and Juliane Hardt \and Sven Kn\"uppel}
\date{\today\footnote{This is an updated pre-print 
version of a ``research letter'' published in the journal Epidemiology
(\url{http://dx.doi.org/10.1097/EDE.0b013e318225c2be}).}}

\setlength{\parindent}{0em}
\setlength{\parskip}{1em}

\begin{document}

\maketitle

Causal diagrams, also known in epidemiology as directed 
acyclic graphs \cite{Pearl2000,GreenlandPR1999},
provide an entirely graphical, yet mathematically
rigorous methodology for minimizing bias in epidemiological studies 
\cite{ShrierP2008,RothmanGL2008}. 
The analysis of causal diagrams can be cumbersome in practice, 
and lends itself well 
to automization by a computer program. Important first steps in this 
regard include the development of the \emph{DAG program} by 
Knüppel \cite{KnueppelS2010} and \emph{dagR} by Breitling 
\cite{Breitling2010}. We are writing to announce the release of 
\emph{DAGitty}, which to our knowledge is the first program 
in the field of epidemiology that provides a graphical user interface 
tailored to draw and analyze causal diagrams. Furthermore, DAGitty overcomes 
some performance obstacles (pointed out by Breitling \cite{Breitling2010})
that affect earlier software when analyzing large diagrams.

The addressed performance issues are two-fold.
First, previous software employed back-tracking algorithms \cite{KnueppelS2010} 
to enumerate and categorize 
all paths from exposure to outcome. This is a reasonable approach for small diagrams, 
but diagrams with tens of variables can already contain millions of paths. 
A full listing of these is of little interest to the human user, 
but can take hours or days to generate.
Instead of a path list, 
DAGitty identifies the subdiagrams involved 
in causal and biasing paths and highlights them in different colours. 
This highlighting algorithm scales to very large diagrams (M Liśkiewicz and J Textor,
submitted manuscript, 2011). It provides to the user 
a vivid impression about how causal and biasing effects ``flow'' in the
diagram, i.e., via which variables and causal arrows these effects are mediated. 

The second problem with previous software 
arose when identifying \emph{minimal sufficient adjustment sets} 
(MSA sets). According to causal diagram theory, 
adjustment for the covariates in an MSA set minimizes bias
when estimating the total effect from exposure to outcome. 
A straightforward approach 
to find MSA sets is checking for each covariate set  
whether it is an MSA set. In a diagram with
50 covariates, this means that $2^{50}$ sets may have to be tested -- 
a 16-digit number which is too large even for computers.  
To identify MSA sets more efficiently, 
we adapted an algorithm proposed recently for a related 
graph-theoretical problem \cite{Takata2010}. This algorithm
is guaranteed to output the list of MSA sets reasonably quickly 
(i.e., in polynomial time per MSA set output). Note however 
that very large or very regularly structured diagrams could in 
theory have millions of different MSA sets. If such diagrams become
practically relevant, further research will be necessary to develop appropriate
computational methods for helping the user to choose appropriate MSA 
sets.  

The described algorithms enable DAGitty's graphical interface to instantly reflect
changes made to the diagram, such as adding a new arrow or inverting an arrow 
with unclear causal direction. Thus, users can interactively assess the effects 
of their modifications to MSA sets and the flow of causal and biasing effects. We anticipate
that these interactive possibilies will be of great help to novice users
in developing an intuition about causal diagram theory, and hope that more
experienced users will find these features useful when comparing 
and deciding between different causal diagrams. 

DAGitty is available under an open-source license allowing free
access, redistribution, and modification. It runs directly in 
most modern web browsers and is available online and for 
download at \url{www.dagitty.net}. The current version of the 
manual is available on arXiv \cite{Textor2015}.

%\bibliographystyle{unsrt}

\begin{thebibliography}{1}

\bibitem{Pearl2000}
Pearl J. 
\newblock {\em Causality: Models, Reasoning, and Inference}.
\newblock Cambridge: Cambridge University Press; 2000.

\bibitem{GreenlandPR1999}
Greenland S, Pearl J, and Robins JM.
\newblock Causal diagrams for epidemiologic research.
\newblock {\em Epidemiology.} 1999;10:37--48.

\bibitem{ShrierP2008}
Shrier I, Platt RW.
\newblock Reducing bias through directed acyclic graphs.
\newblock {\em BMC Medical Research Methodology.} 2008;8:70.

\bibitem{RothmanGL2008}
Glymour MM, Greenland S.
\newblock Causal diagrams.
\newblock In: Rothman KJ, Greenland S, Lash TL, eds. {\em Modern Epidemiology}.
3rd ed. 
\newblock Philadelphia: Lippincott Williams \& Wilkins; 2008:183--209.

\bibitem{KnueppelS2010}
Kn\"uppel S, Stang A.
\newblock {DAG} program: identifying minimal sufficient adjustment sets [Letter]. 
\newblock {\em Epidemiology.} 2010;21:159.

\bibitem{Breitling2010}
Breitling L.
\newblock {dagR}: A suite of {R} functions for directed acyclic graphs [Letter].
\newblock {\em Epidemiology.} 2010;21:586--587.

\bibitem{Takata2010}
Takata K.
\newblock Space-optimal, backtracking algorithms to list the minimal vertex
  separators of a graph.
\newblock {\em Discrete Applied Mathematics.} 2010;158:1660--1667.


\bibitem{Textor2015}
Textor J.
\newblock
Drawing and Analyzing Causal DAGs with DAGitty.
\newblock CoRR abs/1508.04633 (2015). Available at
\url{http://arxiv.org/abs/1508.04633}.


\end{thebibliography}

\end{document}